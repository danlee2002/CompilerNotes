\documentclass[20pt]{article}

\usepackage{lmodern}

\usepackage[document]{ragged2e}
\usepackage{fancyhdr}
\pagestyle{fancy}
\fancyhf{}
\lhead{Compiler Chapter 2 Notes}
\rhead{Daniel Lee}

\rfoot{Page \thepage}
\title{Chapter 2}
\author{Daniel Lee}
\date{May 7th,  2022}
\begin{document}
\maketitle
\newpage
\justify
\section*{Introduction}
    \begin{itemize}
        \item A compiler scans an input of characters and outputs a stream of words labelled by syntatic category
        \item A microsyntax is used to group words that have meaning within the source language  
        \item Some words such as keywords have special meaning, which makes them reserved 
        \item An example of this would be the \textit{while} and \textit{static} keywords in the Java programming language
        \item To recognize keywords, the scanner can either use dictionary lookup or encode keywords directly into microsyntax
        \item The simple lexical structure of programming languages lends itself to efficent scanners
    \end{itemize}
\section*{Recognizing Words}
    \begin{itemize}
        \item When we are parsing words we can view the parsing process as a series of if-else statements or a state machine 
        \item Transition diagrams often provide a simple means of formalizing the abstractions a compiler may need to implement them
        \item S is the finite set of states in the recognizer, alongside with error state $s_e$
        \item $\Sigma$ is the finite alphabet recognized by the recognizer 
        \item $\delta(s,c)$ is the transition function, it maps the value of state s and c,into some state
        \item In state $s_i$ with transition character $c$, the state makes the following transition $s_i \rightarrow_c \delta(s_i,c)$
        \item $s_0 \in S$ refers to initial state
        \item $S_a (S_a \subseteq S)$, is the set of accepting states
    \end{itemize}
    \subsubsection*{Example:}
    $S = \{s_0, s_1, s_2, s_3, .......,s_10,s_e\}\\
    \Sigma = \{ e, h, i, l, n, o, t, w\} \\
    \delta = \\
        \begin{cases}
        s_0 \rightarrow_n s_1, s_0 \rightarrow_w s_6, s_1 \rightarrow_e s_2, s_1 \rightarrow_o s_4, s_2 \rightarrow_w s_3 
        \\s_4 \rightarrow_t s_5, s_6 \rightarrow_h s_7, s_7 \rightarrow_i s_8, s_8 \rightarrow_l s_9, s_9 \rightarrow_e s_{10} 
        \end{cases}\\
        s_0 = s_0\\
        S_A = \{s_3,s_5,s_{10}\}\\
    $

    \newpage
\subsection*{More complex words:}
        \begin{itemize}
            \item For more complex words we can have the state machine accept multiple inputs
            \item We can vastly simplify state machines by using cycles
        \end{itemize}
        \subsubsection*{Practice Problems:}
            \begin{itemize}
                \item Problem 1: A six-character identifier consisting of alphanumeric characters followed by zero to five-alpha numeric characters
                    \begin{itemize}
                        \item $S = \{s_0, s_1,s_e\}$ 
                        \item $\Sigma = {a = \textbf{set of all-alphabet},b = \textbf{set of all alphanumeric}}$
                        \item $s_0 = s_0$
                        \item $\delta = \begin{cases}
                            s_0 \rightarrow_a s_1, s_{1} \rightarrow_b s_{1}
                        \end{cases}$
                        \item $S_A = {s_1}$
                    \end{itemize}
                \item Problem 2: 
                    \begin{itemize}
                        \item $S = \{s_0, s_1,s_2 s_e\}$
                        \item $\Sigma  = {(,)}$
                        \item $s_0 = s_0$
                        \item $S_A = \{s_2\}$
                        \item $\delta = \begin{cases}
                            s_0\rightarrow_{(} s_1, s_1 \rightarrow_{)} s_2, s_2 \rightarrow_{(} s_1 \} 
                        \end{cases}$
                    \end{itemize}
                \item Problem 3: A Pascal comment which consists of \{, zero  or more characters from the alphabet, and closed by \}:
                  \begin{itemize}
                    \item $S = \{s_0, s_1,s_2\}$
                    \item $\Sigma = \{ \},\{,a...z,A...Z,0...9\}$
                    \item $s_0 = s_0 $
                    \item $S_A = \{S_3 \}$
                    \item $\delta = \begin{cases}
                        s_0 \rightarrow_{\{} s_1\\
                        s_1 \rightarrow_{(a...z,A...Z,0...9)} s_1\\
                        s_1 \rightarrow_\{ s_2
                    \end{cases}$
                  \end{itemize}
                  \end{itemize}
    \subsection*{Regular Expression}
                  \begin{itemize} 
                    \item The set of all words accepted by a finite automaton, F, forms a language L(F)
                    \item For any FA, we can describe describe the language using regular expression or RE
                    \item The language consists of single world "new" can be described as RE, new 
                    \newpage
                    \item A language consisting of two words, new or while can be represented as RE new$|$while
                    \item new or not can be represent by RE, n(ew$|$ot)
                    \item Let us consider the example of punctuation marks, a REs for punctuation may appear such as: : ; ? = $>$ ( ) {} []
                    \item Keywords may have an expression such as this: if while this integer instanceof
                    \item more complex RE: 0 $|(0|1|2|3|4|5|6|7|8|9) (0|1|2|3|4|5|6|7|8|9)*$
                    \item \textbf{The following operator is called a kleen operator and indicates there can be zero or more instances of a RE} 
                \end{itemize}

      

\end{document}
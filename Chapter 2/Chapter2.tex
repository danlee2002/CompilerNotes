\documentclass[20pt]{article}

\usepackage{lmodern}

\usepackage[document]{ragged2e}
\usepackage{fancyhdr}
\pagestyle{fancy}
\fancyhf{}
\lhead{Compiler Chapter 2 Notes}
\rhead{Daniel Lee}

\rfoot{Page \thepage}
\title{Chapter 2}
\author{Daniel Lee}
\date{May 7th,  2022}
\begin{document}
\maketitle
\newpage
\justify
\section*{Introduction}
    \begin{itemize}
        \item A compiler scans an input of characters and outputs a stream of words labelled by syntatic category
        \item A microsyntax is used to group words that have meaning within the source language  
        \item Some words such as keywords have special meaning, which makes them reserved 
        \item An example of this would be the \textit{while} and \textit{static} keywords in the Java programming language
        \item To recognize keywords, the scanner can either use dictionary lookup or encode keywords directly into microsyntax
        \item The simple lexical structure of programming languages lends itself to efficent scanners
    \end{itemize}
\section*{Recognizing Words}
    \begin{itemize}
        \item When we are parsing words we can view the parsing process as a series of if-else statements or a state machine 
        \item Transition diagrams often provide a simple means of formalizing the abstractions a compiler may need to implement them
        \item S is the finite set of states in the recognizer, alongside with error state $s_e$
        \item $\Sigma$ is the finite alphabet recognized by the recognizer 
        \item $\delta(s,c)$ is the transition function, it maps the value of state s and c,into some state
        \item In state $s_i$ with transition character $c$, the state makes the following transition $s_i \rightarrow_c \delta(s_i,c)$
        \item $s_0 \in S$ refers to initial state
        \item $S_a (S_a \subseteq S)$, is the set of accepting states
    \end{itemize}
    \subsubsection*{Example:}
    $S = \{s_0, s_1, s_2, s_3, .......,s_10,s_e\}\\
    \Sigma = \{ e, h, i, l, n, o, t, w\} \\
    \delta = \\
        \begin{cases}
        s_0 \rightarrow_n s_1, s_0 \rightarrow_w s_6, s_1 \rightarrow_e s_2, s_1 \rightarrow_o s_4, s_2 \rightarrow_w s_3 
        \\s_4 \rightarrow_t s_5, s_6 \rightarrow_h s_7, s_7 \rightarrow_i s_8, s_8 \rightarrow_l s_9, s_9 \rightarrow_e s_{10} 
        \end{cases}\\
        s_0 = s_0\\
        S_A = \{s_3,s_5,s_{10}\}\\
    $

    \newpage
\subsection*{More complex words:}
        \begin{itemize}
            \item For more complex words we can have the state machine accept multiple inputs
            \item We can vastly simplify state machines by using cycles
        \end{itemize}
        \subsubsection*{Practice Problems:}
            \begin{itemize}
                \item Problem 1: A six-character identifier consisting of alphanumeric characters followed by zero to five-alpha numeric characters
                    \begin{itemize}
                        \item $S = \{s_0, s_1,s_e\}$ 
                        \item $\Sigma = {a = \textbf{set of all-alphabet},b = \textbf{set of all alphanumeric}}$
                        \item $s_0 = s_0$
                        \item $\delta = \begin{cases}
                            s_0 \rightarrow_a s_1, s_{1} \rightarrow_b s_{1}
                        \end{cases}$
                        \item $S_A = {s_1}$
                    \end{itemize}
                \item Problem 2: 
                    \begin{itemize}
                        \item $S = \{s_0, s_1,s_2 s_e\}$
                        \item $\Sigma  = {(,)}$
                        \item $s_0 = s_0$
                        \item $S_A = \{s_2\}$
                        \item $\delta = \begin{cases}
                            s_0\rightarrow_{(} s_1, s_1 \rightarrow_{)} s_2, s_2 \rightarrow_{(} s_1 \} 
                        \end{cases}$
                    \end{itemize}
                \item Problem 3: A Pascal comment which consists of \{, zero  or more characters from the alphabet, and closed by \}:
                  \begin{itemize}
                    \item $S = \{s_0, s_1,s_2\}$
                    \item $\Sigma = \{ \},\{,a...z,A...Z,0...9\}$
                    \item $s_0 = s_0 $
                    \item $S_A = \{S_3 \}$
                    \item $\delta = \begin{cases}
                        s_0 \rightarrow_{\{} s_1\\
                        s_1 \rightarrow_{(a...z,A...Z,0...9)} s_1\\
                        s_1 \rightarrow_\{ s_2
                    \end{cases}$
                  \end{itemize}
                  \end{itemize}
    \subsection*{Regular Expression}
                  \begin{itemize} 
                    \item The set of all words accepted by a finite automaton, F, forms a language L(F)
                    \item For any FA, we can describe describe the language using regular expression or RE
                    \item The language consists of single world "new" can be described as RE, new 
                    \newpage
                    \item A language consisting of two words, new or while can be represented as RE new$|$while
                    \item new or not can be represent by RE, n(ew$|$ot)
                    \item Let us consider the example of punctuation marks, a REs for punctuation may appear such as: : ; ? = $>$ ( ) {} []
                    \item Keywords may have an expression such as this: if while this integer instanceof
                    \item more complex RE: 0 $|(0|1|2|3|4|5|6|7|8|9) (0|1|2|3|4|5|6|7|8|9)*$
                    \item \textbf{The following operator is called a kleen operator and indicates there can be zero or more instances of a RE} 
                \end{itemize}
    \subsection*{Formalizing notes for regular expressions: }
                \begin{itemize}
                    \item Given a regular expression r, we can denote the Language it describes as L(r)
                    \item An RE is made up of 3 operations:
                    \item Alternation: The alternation or union of two sets R and S denoted R$|$S or \{$x|x \in R$ or $x \in S $\}  
                    \item Concatenation: The concatentation of two sets RS contains all strings formed by prependong an element of R onto one from S,
                    or \\\{xy$|x \in R \: and \: y \in S $\}
                    \item Closure: The kleene closue of a set R, denoted by $R^*$ is $\cup_{i = 0}^{\inf} R^i$ is a concantenation of R with itself zero or more times
                    \item Sometimes we can use notation for finite closure if a set is concantenated multiple times: \\
                    $(R|RR|RRR)$
                    \item Positive Closure is Denoted if $RR^*$
                    \begin{itemize}
                        \item If  $ a\in \Sigma$, then a is also an RE denoting set containing only a
			\item If r and s are RES, denoting sets L(r) and L(s) then r $|$ s is a RE denoting the union, or alternation of L(r) and L(s)\\
			Similarly rs is an RE denoting the concatenation of L(r) and L(s)\\
			$r^*$ is an RE denoting Kleene closure of L(r)
			\item $\epsilon$ represents a RE of an empty string
			\item Parentheses have the highest precedence, followed by closure, concatenaton, and alternation
                    \end{itemize}
		    \subsection*{Example:}
		    	\begin{itemize}
				\item Imagine a language in keywords start with a letter in the English alphabet and can be then followed by a sequence of alphanumeric
				character. We can represent the following keyword using RE: \\
				([A...Z]$|[a...z])([A..Z]|([a...z])|([0...9])^*$
				\item Let's consider another example one in which, we are representing unsigned integers:\\
				$(0|([1...9])([0...9])^*$
				\item Unsigned real number: \\
					$(0|([1...9])([1...9])^*)(\epsilon| .[0...9]^*)$
				\item Quoted strings using complement:\\
					A quoted String in a programing language is often composed of a " followed by ", in between these two characters any characters\\
					can appear. In theory we could write a regular expression that contains all the possible characters but this is impractical.
					To circumvent this issue we can use the complement operator:\\
					"$($\^{}"$)$"\\
                \item Comments can often appear in many forms:\\
                    $(//($ \^{}  $\backslash$n $)^*|$
                    $/*($ \^{}* $|$\^{}  */$)^* */)$
                 
                     
                \end{itemize}
			
                    \end{itemize}
            \subsection*{Closure properties of RE:}
                    \begin{itemize}
                        \item Many regular expressions are closed under many operations,i.e if we apply an operation to a RE we get a RE
                        \item Some obvious examples are concatenation, union, and closure
                        \item Imagine we have a collection of regular expressions to describe syntatic categories in a language: $a_0, a_1,...,a_n$
                        \item To describe all the valid words in a language we can use the RE: $a_0 | a_1 | a_2|...|a_n$
                        \item Closure under union suggests that any finite language is a regular language and can be arranged in alternation
                        \item Closure under concantentation also allows us to build complex REs from simpler one's by concatenating them 
                        \item REs are closed under complements
                    \end{itemize}
            \subsection*{Practice Problems: }
                    \begin{itemize}
                        \item Chapter 2: pg 42
                        \item Problem 1:
                        \item $a_0 = [A...Z], a_1 = [a...z], a_2 = [0...9]$
                        \subitem $(a_0|a_1)(a_0a_1a_2)^5$
                        \item Problem 2:
                        \item $a_0 = {}$
                    \end{itemize}



\end{document}
